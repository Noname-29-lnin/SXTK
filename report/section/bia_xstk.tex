\fontsize{13}{15}\selectfont
\section*{Tóm tắt}

Sự phát triển của các nền tảng mạng xã hội không chỉ ảnh hưởng đến hành vi tiêu dùng, trao đổi hàng hóa mà còn ảnh hưởng đến ngành dịch vụ du lịch. Mục đích của nghiên cứu này là xác định các tác động của mạng xã hội đến quyết định đi du lịch của sinh viên Trường Đại học Tôn Đức Thắng. Một cuộc khảo sát trực tuyến bằng bảng câu hỏi đã được sử dụng để thu thập ý kiến của sinh viên Trường Đại học Tôn Đức Thắng, kết quả thu được 101 mẫu khảo sát. Dữ liệu được phân tích bằng cách sử dụng phương pháp thống kê mô tả, đánh giá độ tin cậy của thang đo. Kết quả nghiên cứu cho thấy, có bốn yếu tố của mạng xã hội ảnh hưởng tích cực đến quyết định đi du lịch của sinh viên Trường Đại học Tôn Đức Thắng là: quảng cáo du lịch trên mạng xã hội, độ tin cậy và chất lượng của thông tin, truyền miệng trên mạng xã hội, người có sức ảnh hưởng trên mạng xã hội. Kết quả nghiên cứu nhằm đưa ra những suy nghĩ mang tính khuyến nghị phù hợp cho sinh viên Trường Đại học Tôn Đức Thắng trong sử dụng mạng xã hội hiệu quả để khai thác thông tin về du lịch.

\textbf{Từ khóa:} Mạng xã hội, sinh viên Trường Đại học Tôn Đức Thắng, quyết định đi du lịch.

